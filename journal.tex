\documentclass[11pt,a4paper]{article}

% ====== Packages ======
\usepackage[margin=2.5cm]{geometry}
\usepackage{setspace}
\usepackage{hyperref}
\usepackage{enumitem}
\usepackage{tabularx}
\usepackage{longtable}
\usepackage{array}
\usepackage{booktabs}
\usepackage{todonotes}

\newcommand\mytodo[2][]{\todo[inline, color=green!20, caption={2do}, #1]{
\begin{minipage}{\textwidth-4pt}#2\end{minipage}}}

\setstretch{1.15}

% ====== Metadata ======
\title{COMP34111 - Journal Report}
% \author{%
%   % Replace with your details
%   Your Name (Student ID: 12345678)\\
%   Group: X
% }
% \date{Academic Year 2025--26}
\date{}

\begin{document}
\maketitle

\section*{Submission Information}
\begin{itemize}[leftmargin=1.5em]
  \item \textbf{Course:} COMP34111
  \item \textbf{Academic Year:} 2025--26
  \item \textbf{Student Name:} \mytodo{Benjamin H Kidd}
  \item \textbf{Student ID:} \mytodo{11019777}
  \item \textbf{Group:} \mytodo{24}
  \item \textbf{Deadline:} 19th December, 2pm
  \item \textbf{Note:} Each student must submit their own copy. No anonymous submissions.
\end{itemize}

\bigskip

\newpage
% ==========================================================
% PART 1 – INDIVIDUAL CONTRIBUTION & REFLECTION (20%)
% ==========================================================
\section{Individual Contribution and Reflection (20\%)}
% Explain what you have done and reflect on how the group work is going.
% Remember: other group members will read this.
% Focus on:
% - What you personally did
% - How you collaborated
% - What went well / not so well
% - What you plan to improve

\subsection{Summary of My Contributions}
% Replace the text below with your own content.
\mytodo{In this section, describe clearly and honestly what \textbf{you} have done for the project so far.  Mention specific tasks, responsibilities, and any deliverables you produced.}
\mytodo{
}

\subsection{Reflection on Group Work}
\mytodo{
\begin{itemize}[leftmargin=1.5em]
    \item None
\end{itemize}
}

\subsection{Personal Development}
% Optional but useful reflection sub-section.
\mytodo{
Comment on:
\begin{itemize}[leftmargin=1.5em]
    \item None
\end{itemize}
}

% ==========================================================
% PART 2 – GROUP MEETING LOG (30%)
% ==========================================================
\section{Group Meeting Log (30\%)}
% This section should record EVERY group meeting.
% For each meeting include:
% - Time, date, location
% - Members present
% - Apologies for absence
% - List of tasks agreed (who does what, and by when)
% - Brief rationale for decisions

% You can either:
% 1) Duplicate the "Meeting Template" subsection below for each meeting, OR
% 2) Use the long table that follows for a compact overview.

\subsection*{Log}

\mytodo{

\subsubsection*{Meeting \#1}
\textbf{Date:} 21/11/2025 \\
\textbf{Time:} 10:00--12:30 \\
\textbf{Location:} Room (Tootil)

\paragraph{Members Present}
\begin{itemize}[leftmargin=1.5em]
  \item Ben (11019777)
  \item Naddy (11011468)
  \item David (11054893)
  \item Daniel (11017362)
  % etc.
\end{itemize}

\paragraph{Apologies for Absence}
\begin{itemize}[leftmargin=1.5em]
  \item None
  % If no apologies, write: None.
\end{itemize}

\paragraph{Tasks Agreed}:
% For clarity, use a table.

\begin{tabularx}{\textwidth}{@{} l l l X @{}}
\toprule
\textbf{Task ID} & \textbf{Responsible} & \textbf{Due Date} & \textbf{Description} \\
\midrule
T1 & Ben & 12/12 & Utilize an Alpha Zero Approach, using MCTS \& Deep Learning. \\
T2 & Daniel & 09/12 & Evalutation function. \\
T3 & Daniel & 12/12 & Alpha Beta. \\
T4 & Naddy \& David  & 12/12 & Manual MCTS (Policy Selection, etc). \\
T5 & David & Explore evolutionary stategies for Deep learning  \\
% Add more rows as needed
\bottomrule
\end{tabularx}

\paragraph{Rationale for Decisions}
\begin{itemize}[leftmargin=1.5em]
  \item Separating individual approaches allowed for the most parallelisation of work.
  \item We felt that within the unit, the main approaches available were AlphaBeta (MiniMax), MCTS and AlphaGo (AlphaZero), which could be expanded on using other approaches. We decided to attempt all of these, and then evaluate which performed the best at the end.
  \item Tasks were divided based on interest and location. David and Naddy live together so it made sense for them to share MCTS. This was considered the largest task, and most important as it could utilise approaches from others (e.g. evalutation functions, trained neural networks).
\end{itemize}
}

\bigskip

\mytodo{

\subsubsection*{Meeting \#2}
\textbf{Date:} 25/11/2025 \\
\textbf{Time:} 13:00--15:00 \\
\textbf{Location:} Room (Tootil)

\paragraph{Members Present}
\begin{itemize}[leftmargin=1.5em]
  \item Ben (11019777)
  \item Naddy (11011468)
  \item David (11054893)
  \item Daniel (11017362)
  % etc.
\end{itemize}

\paragraph{Apologies for Absence}
\begin{itemize}[leftmargin=1.5em]
  \item None
  % If no apologies, write: None.
\end{itemize}

\paragraph{Tasks Agreed}
None (Collectively wrote and submitted the group proforma).
}

\bigskip

\mytodo{

\subsubsection*{Meeting \#3}
\textbf{Date:} 28/11/2025 \\
\textbf{Time:} 10:00--13:00 \\
\textbf{Location:} Room (Tootil)

\paragraph{Members Present}
\begin{itemize}[leftmargin=1.5em]
  \item Ben (11019777)
  \item Naddy (11011468)
  \item Daniel (11017362)
  % etc.
\end{itemize}

\paragraph{Apologies for Absence}
\begin{itemize}[leftmargin=1.5em]
  \item David (11054893), Another coursework submission (NLP).
  % If no apologies, write: None.
\end{itemize}

\paragraph{Tasks Agreed}:
% For clarity, use a table.

\begin{tabularx}{\textwidth}{@{} l l l X @{}}
\toprule
\textbf{Task ID} & \textbf{Responsible} & \textbf{Due Date} & \textbf{Description} \\
\midrule
T5 & Naddy & 30/11 & Research relevant papers on how to implement MCTS in Hex. \\
T6 & Daniel & 02/12 & Create a visualiser for the evaluation function. \\
T6 & David & 02/12 & Research relevant papers and start work towards an implementation of a genetic stragegy \\
% Add more rows as needed
\bottomrule
\end{tabularx}

\paragraph{Rationale for Decisions}
\begin{itemize}[leftmargin=1.5em]
  \item There are plenty of resources available from researchers who have optimised Hex agents before. Their findings will help with making the algorithm faster.
  \item It is difficult to identify if what the evaluation function is doing makes sense or not, so we should have an easy way to see what it thinks.
  \item There are several papers on implementing evolutionary strategies to teach neural networks to play Hex and other similar games. While this approach was not widely used, it did look promising as an alternative which may outperform more simple agents.
\end{itemize}
}

\bigskip

\mytodo{

\subsubsection*{Meeting \#4}
\textbf{Date:} 02/12/2025 \\
\textbf{Time:} 13:00--15:00 \\
\textbf{Location:} Room (Tootil)

\paragraph{Members Present}
\begin{itemize}[leftmargin=1.5em]
  \item Ben (11019777)
  \item Naddy (11011468)
  \item David (11054893)
  \item Daniel (11017362)
  % etc.
\end{itemize}

\paragraph{Apologies for Absence}
\begin{itemize}[leftmargin=1.5em]
  \item None
  % If no apologies, write: None.
\end{itemize}

\paragraph{Tasks Agreed}:
% For clarity, use a table.

\begin{tabularx}{\textwidth}{@{} l l l X @{}}
\toprule
\textbf{Task ID} & \textbf{Responsible} & \textbf{Due Date} & \textbf{Description} \\
\midrule
TX & All & 07/12 & Create and fill out our presentation. \\
TX & Naddy & 09/12 & Run some experiments with the MCTS implementation. \\
TX & Daniel & 09/12 & Tune heuristics used in evaluation function. Determine if alpha beta is worthwhile spending time on. \\
TX & Ben & 09/12 & Implement prioritisation to speed up winning games. \\
TX & David & 09/12 & Implement simple neural agent and look into strategies to evolve it \\

% Add more rows as needed
\bottomrule
\end{tabularx}

\paragraph{Rationale for Decisions}
\begin{itemize}[leftmargin=1.5em]
  \item Our presentation is coming up soon and we want to get it done sooner rather than later, so it won't get in the way of making our models.
  \item The presentation requires some experimental results so ideally the vanilla MCTS implementation should be working and winning before then.
  \item The current heuristics were very simple, and we need a way to score patterns. Even so, alpha beta might not be promising enough.
  \item The AlphaZero approach was winning consistently, but was taking almost full boards to win.
\end{itemize}
}

\bigskip

\mytodo{

\subsubsection*{Meeting \#5}
\textbf{Date:} 05/12/2026 \\
\textbf{Time:} 10:00--12:00 \\
\textbf{Location:} Room (Tootil)

\paragraph{Members Present}
\begin{itemize}[leftmargin=1.5em]
  \item Ben (11019777)
  \item Naddy (11011468)
  \item David (11054893)
  \item Daniel (11017362)
  % etc.
\end{itemize}

\paragraph{Apologies for Absence}
\begin{itemize}[leftmargin=1.5em]
  \item None
  % If no apologies, write: None.
\end{itemize}

\paragraph{Tasks Agreed}
None (Collectively wrote the group powerpoint).
}

\bigskip

\mytodo{

\subsubsection*{Meeting \#6}
\textbf{Date:} 08/12/2025 \\
\textbf{Time:} 13:00--15:00 \\
\textbf{Location:} Hybrid, Kilburn and Discord

\paragraph{Members Present}
\begin{itemize}[leftmargin=1.5em]
  \item Ben (11019777)
  \item Naddy (11011468)
  \item David (11054893)
  \item Daniel (11017362)
  % etc.
\end{itemize}

\paragraph{Apologies for Absence}
\begin{itemize}[leftmargin=1.5em]
  \item None
  % If no apologies, write: None.
\end{itemize}

\paragraph{Tasks Agreed}:
% For clarity, use a table.

\begin{tabularx}{\textwidth}{@{} l l l X @{}}
\toprule
\textbf{Task ID} & \textbf{Responsible} & \textbf{Due Date} & \textbf{Description} \\
\midrule
TX & All & 09/12 & Script and rehearse the presentation. \\
TX & Naddy & 11/12 & Look into implementing RAVE or other optimisations for MCTS. \\
TX & Daniel & 11/12 & Create bitmasks for common patterns to speed up recognition of said patterns. \\
TX & Ben & 15/12 & Implement a minimax checker for AlphaZero to help with game endings. \\
% Add more rows as needed
\bottomrule
\end{tabularx}


\paragraph{Rationale for Decisions}
\begin{itemize}[leftmargin=1.5em]
  \item We need to ensure we're within the time limit of the presentation to get everyone's input.
  \item MCTS was not quick enough at finding optimal moves on a board as big as 11x11.
  \item Evaluation was slow when ran with alpha beta minimax, bit manipulation operations would speed it up significantly.
  \item Game endings were struggling to be found, sometimes with the AI having a winning board state but never playing it. Minimax used at the start of the move to check up to a given depth (which increases based on the number of moves played to keep compute power reasonable), playing winning moves if found, or blocking moves if opponent winning moves found. If no certain win/loss moves are found, normal AlphaZero is used.
\end{itemize}
}

\bigskip

\mytodo{

\subsubsection*{Meeting \#7}
\textbf{Date:} 09/12/2025 \\
\textbf{Time:} 13:00--15:00 \\
\textbf{Location:} Room (Tootil)

\paragraph{Members Present}
\begin{itemize}[leftmargin=1.5em]
  \item Naddy (11011468)
  \item David (11054893)
  \item Daniel (11017362)
  % etc.
\end{itemize}

\paragraph{Apologies for Absence}
\begin{itemize}[leftmargin=1.5em]
  \item Ben (11019777), CGI Assessment Centre.
  % If no apologies, write: None.
\end{itemize}

\paragraph{Tasks Agreed}:
% For clarity, use a table.

\begin{tabularx}{\textwidth}{@{} l l l X @{}}
\toprule
\textbf{Task ID} & \textbf{Responsible} & \textbf{Due Date} & \textbf{Description} \\
\midrule
TX & Daniel & 11/12 & Explore any other potential approaches. \\
TX & Naddy & 14/12 & Run optimised MCTS implementation against other models. \\
TX & Ben & 15/12 & Utilise Supervised Learning (AlphaGo style) for initial learning. \\
% Add more rows as needed
\bottomrule
\end{tabularx}


\paragraph{Rationale for Decisions}
\begin{itemize}[leftmargin=1.5em]
  \item This will be done to find the MCTS implementation's weaknesses, and to give guidance on what to focus a heuristic on.
  \item Minimax approach not promising, explore any other potential approaches we havent considered.
  \item While the model was learning well, it would take lots of games to get strong. Utilise known good games to train the AI initially, hoping it will pick up on some good initial patterns/weights.
\end{itemize}
}

\bigskip

\mytodo{

\subsubsection*{Meeting \#8}
\textbf{Date:} 12/12/2025 \\
\textbf{Time:} 10:00--13:00 \\
\textbf{Location:} Room (Tootil)

\paragraph{Members Present}
\begin{itemize}[leftmargin=1.5em]
  \item Ben (11019777)
  \item Naddy (11011468)
  \item David (11054893)
  \item Daniel (11017362)
  % etc.
\end{itemize}

\paragraph{Apologies for Absence}
\begin{itemize}[leftmargin=1.5em]
  \item None.
  % If no apologies, write: None.
\end{itemize}

\paragraph{Tasks Agreed}:
% For clarity, use a table.

\begin{tabularx}{\textwidth}{@{} l l l X @{}}
\toprule
\textbf{Task ID} & \textbf{Responsible} & \textbf{Due Date} & \textbf{Description} \\
\midrule
TX & Naddy & 16/12 & Research further optimisations for the MCTS implementation and add them.  \\
TX & Daniel & 16/12 & Apply h-search and minimax optimisations to MCTS approach.
TX & Ben & 18/12 & Utilise existing KataHex model since the coursework allows it, to avoid extensive training being required. \\
% Add more rows as needed
\bottomrule
\end{tabularx}


\paragraph{Rationale for Decisions}
\begin{itemize}[leftmargin=1.5em]
  \item The MCTS implementation is performing decently, but it still takes too long per turn to reach that point. May considering moving away from python.
  \item H search would help optimise MCTS expansion policy.
  \item TX. While AlphaZero was learning correctly, it would need massive amounts of computing power to train to a decent level. Sicne the discussion board says we can use pre-trained models, I implement a bot based on the existing KataHex model.
\end{itemize}
}

\bigskip

\mytodo{

\subsubsection*{Meeting \#9}
\textbf{Date:} 15/12/2025 \\
\textbf{Time:} 13:00--15:00 \\
\textbf{Location:} Room (Tootil)

\paragraph{Members Present}
\begin{itemize}[leftmargin=1.5em]
  \item Ben (11019777)
  \item Naddy (11011468)
  \item David (11054893)
  \item Daniel (11017362)
  % etc.
\end{itemize}

\paragraph{Apologies for Absence}
\begin{itemize}[leftmargin=1.5em]
  \item None.
  % If no apologies, write: None.
\end{itemize}

\paragraph{Tasks Agreed}:
% For clarity, use a table.

\begin{tabularx}{\textwidth}{@{} l l l X @{}}
\toprule
\textbf{Task ID} & \textbf{Responsible} & \textbf{Due Date} & \textbf{Description} \\
\midrule
TX & Ben (1101977) & 18/12 & Complete their journal version \\
TX & Naddy (11011468) & 18/12 & Complete their journal version \\
TX & David (11054893) & 18/12 & Complete their journal version \\
TX & Daniel (11017362) & 18/12 & Complete their journal version \\
TX & Ben & 18/12 & In the same way as with the AlphaZero approach, add Minimax for KataHex. \\
\bottomrule
\end{tabularx}


\paragraph{Rationale for Decisions}
\begin{itemize}[leftmargin=1.5em]
  \item T12. Same rationale as T6. 
  \item Decision made to submit XXXX bot
\end{itemize}
}

\bigskip

\mytodo{

\subsubsection*{Meeting \#10}
\textbf{Date:} 18/12/2025 \\
\textbf{Time:} 12:00--14:00 \\
\textbf{Location:} Room (Tootil)

\paragraph{Members Present}
\begin{itemize}[leftmargin=1.5em]
  \item Ben (11019777)
  \item Naddy (11011468)
  \item David (11054893)
  \item Daniel (11017362)
  % etc.
\end{itemize}

\paragraph{Apologies for Absence}
\begin{itemize}[leftmargin=1.5em]
  \item None.
  % If no apologies, write: None.
\end{itemize}

\paragraph{Tasks Agreed}
None (Collectively finalised project).
}

\bigskip

% ---------- Optional: Compact Summary Table of All Meetings ----------
\subsection{Compact Summary of Meetings (Optional)}
% You can use this as an overview of all meetings, in addition to the detailed templates above.

\setlength{\extrarowheight}{2pt}
\begin{longtable}{@{}p{2.8cm} p{3.5cm} p{3cm} p{5.5cm}@{}}
\caption{Summary of Group Meetings} \\
\toprule
\textbf{Date \& Time} & \textbf{Location} & \textbf{Members Present} & \textbf{Main Outcomes / Tasks} \\
\midrule
\endfirsthead

\toprule
\textbf{Date \& Time} & \textbf{Location} & \textbf{Members Present} & \textbf{Main Outcomes / Tasks} \\
\midrule
\endhead

% ---- Example rows (delete or overwrite) ----
21/11/2025, 10:00 & Room & Ben, Naddy, David, Daniel & Defined Individual Approaches. \\
25/11/2025, 13:00 & Room & Ben, Naddy, David, Daniel & Started work on current approaches, discussed methods of implementation. \\
28/11/2025, 10:00 & Room & Ben, Naddy, David, Daniel & Submitted Proforma and continued work on current approaches, updated teammates on progress. \\
2/12/2025, 13:00 & Room & Ben, Naddy, Daniel & Continued work on current approaches, updated teammates on progress. Considering removing Alpha Beta from scope.\\
05/12/2025, 10:00 & Room & Ben, Naddy, David, Daniel & Created Powerpoint. \\
08/12/2025, 13:00 & Hybrid & Ben, Naddy, David, Daniel & Finalised and Practiced Powerpoint \\
09/12/2025, 13:00 & Room & Naddy, David, Daniel & Continued work on current approaches, updated teammates on progress. Decided to remove Alpha Beta from scope.\\
12/12/2025, 10:00 & Room & Ben, Naddy, David, Daniel & Continued work on current approaches, updated teammates on progress. Started work on KataHex approach. \\
15/12/2025, 13:00 & Hybrid & Ben, Naddy, David, Daniel & Compared bots and finalised log. Added final MiniMax improvements to KataHex agent. \\
18/12/2025, 12:00 & Hybrid & Ben, Naddy, David, Daniel & Finalised everything for submission \\
% Add more rows as needed

\bottomrule
\end{longtable}

% ==========================================================
% PART 3 – BOT METHOD & SIGNATURES (50%)
% ==========================================================
\section{Bot Method and Unique Strengths (50\%)}
% This section is about the QUALITY of the approach you attempted.
% State:
% - What method the bot used
% - Design choices and architecture
% - Unique aspects that made it strong and powerful

\subsection{Method Overview}
\mytodo{
Describe what method your bot used. For example:
\begin{itemize}[leftmargin=1.5em]
  \item Algorithm(s) or model(s) used
  \item High-level architecture
  \item How the components interact
\end{itemize}
}
\paragraph{
  TODO
}

\subsection{Technical Details}
\mytodo{
Provide more detail here, such as:
\begin{itemize}[leftmargin=1.5em]
  \item Input representation and features
  \item Training, tuning, or rule design
  \item Any libraries, frameworks, or tools used
  \item Evaluation strategy and results
\end{itemize}
}
\paragraph{
  TODO
}

\subsection{Unique Aspects and Strengths}
\mytodo{
Explain what made your bot \emph{strong and powerful}:
\begin{itemize}[leftmargin=1.5em]
  \item Novel ideas or design decisions
  \item Strengths compared to a baseline or naive approach
  \item Robustness, efficiency, scalability, etc.
\end{itemize}
}
\paragraph{
  TODO
}

\subsection{Limitations and Possible Improvements}
\mytodo{
Briefly acknowledge:
\begin{itemize}[leftmargin=1.5em]
  \item Known weaknesses or failure cases
  \item Improvements you would make with more time
\end{itemize}
}
\paragraph{
  TODO
}

% ---------- Signatures ----------
\subsection{Group Member Signatures}
Each member of the group should sign this section with their name and student ID.

\vspace{1em}

\begin{tabularx}{\textwidth}{@{}p{4cm} p{4cm} X @{}}
\toprule
\textbf{Name} & \textbf{Student ID} & \textbf{Signature / Confirmation} \\
\midrule
\mytodo{Ben} & \mytodo{11019777} & I confirm the above description of the method and my contribution. \\
\mytodo{Naddy} & \mytodo{11011468} & I confirm the above description of the method and my contribution. \\
\mytodo{David} & \mytodo{11054893} & I confirm the above description of the method and my contribution. \\
\mytodo{Daniel} & \mytodo{11017362} & I confirm the above description of the method and my contribution. \\
% Add more rows as needed
\bottomrule
\end{tabularx}

% ==========================================================
% APPENDIX – FURTHER RULES (OPTIONAL COPY)
% ==========================================================
\newpage
\appendix
\section{Further Rules (For Reference Only)}
\begin{enumerate}[leftmargin=1.5em]
  \item No anonymous submissions.
  \item Every group member should attend every meeting. If you cannot attend (illness, job interview, etc.) let your group teammates know as soon as possible.
  \item If your group is not functioning or someone is not engaging, responding to communications or attending meetings, please let Mingfei Sun (\href{mailto:mingfei.sun@manchester.ac.uk}{mingfei.sun@manchester.ac.uk}) know as soon as possible.
\end{enumerate}

\end{document}
